\textbf{Adversarial camera and LiDAR-based attacks AVs}
Attacks in perception sensors can be divided into two categories, camera-based attack, and object-based attack. The
camera-based attack methods\cite{7, 9, 23} propose to hide the
objects to be detected by adding adversarial patches. The attacker
can apply different interference methods to enhance the
robustness so that the objects won’t be detected from varying
observation angles and distances. This camera-based attack
aims to change the texture of the object\cite{msf-adv}. The Lidar-based
attack methods\cite{4, 6, 19, 25} propose to spoof the LiDAR with
injecting laser\cite{6}, finding vulnerable LiDAR detection locations\cite{25} or changing the shape of the 3D objects\cite{4}. This
kind of attack can fool the LiDAR object detection mechanism,
but it’s hard to spoof cameras as it aims to change the shapes
instead of the texture of the object\cite{msf-adv}. 
In these works, to mislead the neural network, some outstanding patterns are generated
to cause the model to have a tendency towards specific outputs.

\textbf{Defense towards the adversarial camera and LiDAR-based
attacks in AVs}
Defenses against these adversarial perception attacks also
fall into two types. One kind of defense\cite{if-defense, 22, 24} aims to
detect and recover the corrupted objects before they’re sent
to the detection algorithm. The authors reconstruct the objects
with implicit functions\cite{if-defense} or denoising and upsampling\cite{24}.
Although these methods can achieve a good recovering rate,
they focus on either camera-based attacks or object-based
attack. The other kind of defense aims to fuse multiple sensors\cite{10, 15, 16, 21} to avoid the spoofed sensor guiding the
detection output. These Multiple Sensor Fusion (MSF) algorithms integrate the image and LiDAR feature map 
strategically to rely on the unattacked sensors.

